\documentclass[a4paper,12pt]{article}

\usepackage[english,francais]{babel}
\usepackage[latin1]{inputenc}
%\usepackage[utf8]{inputenc}
\usepackage[T1]{fontenc}
\usepackage{amsmath}
\usepackage{amsfonts}
\usepackage{amssymb}

\newcommand{\df}[1]{{\emph{#1}}}

\newcommand{\C}{\mathbb{C}}
\newcommand{\N}{\mathbb{N}}
\newcommand{\R}{\mathbb{R}}
\newcommand{\Z}{\mathbb{Z}}
\newcommand{\Q}{\mathbb{Q}}

\newcommand{\Hom}{\operatorname{Hom}}
\newcommand{\Aut}{\operatorname{Aut}}
\newcommand{\End}{\operatorname{End}}
\newcommand{\Mat}{\operatorname{Mat}}
\newcommand{\id}{\operatorname{Id}}

\newcommand{\im}{\operatorname{im}}
\newcommand{\tr}{\operatorname{tr}}

\newcommand{\s}{\mathfrak{S}}

\newcommand{\isom}{\stackrel{\sim}{\longrightarrow}}

\newcommand{\inj}{\hookrightarrow}
\usepackage{stmaryrd}

\newcommand{\SL}{\text{SL}}
\newcommand{\GL}{\text{GL}}
\newcommand{\M}{\text{M}}
\newcommand{\mat}[4]{\begin{pmatrix}
#1 & #2 \\
#3 & #4
\end{pmatrix}}
\newcommand{\eps}{\varepsilon}

%\newtheorem{enonce}{Exercice}
%\newenvironment{exo}[0]{\begin{enonce}\rm}{\smallskip\end{enonce}}
\newcounter{question}
\newcounter{sousquestion}
\newtheorem{enonce}{Exercice}
\newenvironment{exo}[0]{\begin{enonce}{\bf ---}\rm\setcounter{question}{1}}{\end{enonce}}
\newcommand{\quest}{{\setcounter{sousquestion}{1}\vspace{0.1cm}\bf \arabic{question}.\hspace{0.1cm}}\addtocounter{question}{1}}
\newcommand{\sousq}{{\hspace{0.5cm}\vspace{0.1cm}\bf \alph{sousquestion}.\hspace{0.1cm}}\addtocounter{sousquestion}{1}}



\begin{document}

\noindent \'ENS de Lyon\hfill \\
2018-2019\hfill Agr�gation\
\smallskip
\begin{center}
{\sc
Le�on 120 : Anneaux $\Z/n\Z$. Applications.}
\end{center}
\smallskip


\begin{exo}
Soit $n\in \N^*$. D�terminer les id�aux de $\Z/n\Z$. L'anneau $\Z/n\Z$ est-il principal ?
\end{exo}

\begin{exo}
Donner l'�criture d�cimal de $\frac{5}{7}$.
\end{exo}

\begin{exo}
Soient $n,n'\in \N$. D�terminer tous les morphismes d'anneaux de $\Z/n\Z$ dans $\Z/n'\Z$.
\end{exo}

\begin{exo}
R�soudre $x^2-4x+6$ dans $\Z/12 \Z$.
\end{exo}

\begin{exo}
 Soit $n\in \N^*$. Combien y a-t-il d'idempotents (i.e de $x$ tels que $x^2=x$) dans $\Z/n\Z$.
\end{exo}

\begin{exo}
R�soudre $x^2+y^2=7 z^2$, d'inconnue $(x,y,z)\in \Z^3$.
\end{exo}

\begin{exo}
Montrer que l'equation $(x^2-2)(x^2-3)(x^2-6)$ d'inconnue $x$ admet une solution dans $\mathbb{F}_p$ pour tout nombre premier $p$, mais qu'elle n'admet aucune solution sur $\Z$.
\end{exo}

\begin{exo}
Soit $n\in \N_{\geq 3}$.

\quest Montrer que pour tout $x\in (\Z/2^n\Z)^\times$, $x^{2^{n-2}}=1$. Le groupe $(\Z/2^n\Z )\times$ est-il cyclique ?

\quest Montrer que $5^{2^n}\equiv 1+2^{n+2}[2^{n+3}]$.

\quest On identifie $\Z/2\Z$ � $\{+,-\}$. Soit $\phi:\Z/2\Z\times \Z/2^{n-2}\Z\rightarrow (\Z/2^n\Z)^\times$ d�finie par $\phi((\epsilon,k))=\epsilon 5^k$. Montrer que $\phi$ est bien d�finie et que c'est un isomorphisme.

\quest Soit $p$ un nombre premier impair. Montrer que pour tout $n\in \N$, il existe $\ell$ premier � $p$ tel que $(1+p)^{p^n}=1+\ell p^{n+1}$. Quelle est la structure de $(\Z/p^k\Z)^\times$.
\end{exo}

\begin{exo}
Soit $n\in \N^*$. Montrer que $\mathrm{Aut}_{\mathrm{Gr}}(\Z/n\Z)\simeq (\Z/n\Z)^\times$. D�terminer $\mathrm{Aut}_{\mathrm{Gr}}((\Z/n\Z)^\times)$.
\end{exo}

\begin{exo} (RSA, voir Demazure, Cours d'alg�bre, page 59)
Soient $p,q$ des nombres premiers distincts. Soit $n=pq$.

\quest D�terminer $\varphi(n)$, o� $\varphi$ est l'indicatrice d'Euler.

\smallskip 

Soit  $e\in \llbracket 1, \varphi(n)\rrbracket$ tel que $e\wedge \varphi(n)=1$. 

\quest Soit $M\in \llbracket 1,n\rrbracket$ tel que $C\wedge n=1$. Soit $C=M^e$. Comment d�terminer $M$ � partir de $C$ ? 
\end{exo}

\end{document}
